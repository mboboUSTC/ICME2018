\section{EXPERIMENTS}
In this section, we will evaluate our framework with different parameters and compare our method with several state-of-the-art methods on segmenting the synaptic cleft region in electron micrographs.

~\noindent\textbf{Dataset}
Synaptic images are obtained by cryo-electron tomography (CET), from which we can directly observe a native environment of synaptic structures in a high resolution (about $1500\times 1500$).
\xj{native environment or natural environment?}
%
And localizing the accurate contour of a target region in such a crowded and natural environment is significantly challenging, due to \change{the high nose} and low signal-to-noise ratio.
In this paper, our goal is to extract the synaptic cleft region, which is adjacent to a synapse and receives neurotransmitter molecules from another synapse.
Especially, only the cleft between two synapse, whose width is about $20\sim30$ nm ($40\sim70$ pixels in our electron micrographs), might by our plausible synaptic cleft, as shown in Fig.~\xj{no?}.
%
For evaluating our method, we build a synaptic electron micrographs dataset, consisting $400$ synaptic images for training and $159$ images for testing.
All the image are observed in raw resolution and labeled by biomedical experts.

~\noindent\textbf{Implementation Details}
The training strategy of our DeepLab mode follows \cite{Chen2016a}.
For such a large resolution, we crop a $481\times 481$ region \cite{Chen2016} from the original image as the input to DeepLab.
In order to avoid over fitting, the training dataset is augmented by flipping and rotation, finally containing $19200$ images.

During curve evolving, $\alpha$, $\beta$, $\kappa$ and $\gamma$ are respectively set as $0.2$, $0.2$, $0.3$ and $1$, which can give a best performance in our dataset.
For synchronous growing, $\rho$ is set as $0.25$ and $\tau_1$, $\tau_2$ are $-\frac{\sqrt{2}}{2}$, $\frac{\sqrt{2}}{2}$ to constraint the direction of a new piece of contour to deviating $[-\frac{\pi}{4},\frac{\pi}{4}]$ from last growing direction.
And when $d^{t}$ is beyond the range of $[40,80]$, the growing is terminated.

We use two metrics \cite{Cheng2017} to evaluate our method on segmentation task:
a) pixel accuracy, which evaluates the percentage of positive true pixels over the whole pixels;
b) pixel intersection-overunion (IOU) averaged across different classes.

\begin{figure}[t]
\begin{minipage}[b]{1.0\linewidth}
  \centering
 \centerline{\epsfig{figure=figs/FigGVFa.pdf,width=8.5cm}}
  \centerline{(a) Tension field by GVF.}\medskip
\end{minipage}
\begin{minipage}[b]{1.0\linewidth}
 \centerline{\epsfig{figure=figs/FigGVFb.pdf,width=8.5cm}}
  \centerline{(a) Tension field by normal force.}\medskip
\end{minipage}
%
\caption{Several examples of tension field obtained by GVF and normal force. Each column corresponds to one example.}
\label{fig:gvf}
\end{figure}
~ \noindent\textbf{Superiority of Eq.~\ref{Eq:update}.}
In this part, we visualize the mentioned deficiencies in Sec.~\ref{sec:curve_evolving} and show the superiority of our updating strategy of Eq.~\ref{Eq:update}.
Firstly, in some flat region with small gradients $\mathbf{f}$, the external tension is too weak to drive $\mathbf{v}(s)$ to move against to internal tension, which gives a high demand on initial curve to be away from those flat regions (first example in Figure~\ref{fig:gvf}).
Secondly, the external tension in some noisy regions will be gyrate, which easily trap some $\mathbf{v}(s)$.
It is frustrating that although the internal tension may pull the trapped point out sometimes, $\mathbf{f}$ will soon trap it again.
However using our evolving strategy , once $\mathbf{v}(s)$ is pulled away from the noisy region with small $E_{ext}$, our external tension with fixed normal direction will soon push it away (second example in Figure~\ref{fig:gvf}).
Thirdly, since the gradients $\mathbf{f}$ near to contours are usually large (Figure~\ref{fig:gvf}), $\mathbf{v}(s)$ easily cross the optimal positions by over-huge tension.
And our external tension are controlled by $E_{ext}$, which will soon vanish in contour region and make the updating more robust (third example in Figure~\ref{fig:gvf}).

~\noindent\textbf{Results}
We evaluate several state of the art segmentation methods with our proposed model in segmenting synaptic cleft region, including FCN \cite{Long2015}, U-net \cite{Ronneberger2015}, DeepLab(ResNet-101) \cite{Chen2016a} and PSPNet \cite{Zhao2016}.
Several improved strategies have been tried to give a best performance, including weighted loss function, transfer learning and post-processing.
Table~\ref{tab:report1} reports the pixel accuracy and mean IOU of different methods and Figure~ represents some results from different methods.
From the Table, we can find that our contour growing algorithm can significantly improve the performance of single CNN based model.
Observed from Figure, our experiments prove the deficiency of CNN in localizing accurate contours, which can be remedied by our contour growing algorithm.
Especially due to the deeper architecture of PSPNet \cite{Zhao2016}, it gives a better performance over other sole CNN models.
\begin{table}[t]
\begin{center}
\caption{Table caption} \label{tab:report1}
\begin{tabular}{|c|c|c|}
  \hline
  % after \\: \hline or \cline{col1-col2} \cline{col3-col4} ...
   & Pixel Accu. & Neab IOU
  \\
  \hline
  FCN & Cell 2 & Cell 3 \\
  Unet & Cell 5 & Cell 6 \\
  DeepLab(ResNet-101) & Cell 5 & Cell 6 \\
  DeepLab(ResNet-101)+CRF & Cell 5 & Cell 6 \\
  PSPNet & Cell 5 & Cell 6 \\
  Ours & Cell 5 & Cell 6 \\
  \hline
\end{tabular}
\end{center}
\end{table}

Furthermore, we replace our pre-segmentation model with other CNN based methods to explore the robustness of our model to different initial curve.
Table gives the results, where we can find that there is no obvious improvement by using better pre-segmenting model.
It demonstrates that so long as the correct location of synaptic cleft region is provided, our model can localize the whole contours of target region.

\begin{table}[t]
\begin{center}
\caption{Table caption} \label{tab:report1}
\begin{tabular}{|c|c|c|}
  \hline
  % after \\: \hline or \cline{col1-col2} \cline{col3-col4} ...
   & Pixel Accu. & Neab IOU
  \\
  \hline
  FCN$+$contour growing & Cell 2 & Cell 3 \\
  Unet$+$contour growing & Cell 5 & Cell 6 \\
  DeepLab$+$contour growing & Cell 5 & Cell 6 \\
  Resnet$+$contour growing & Cell 5 & Cell 6 \\
  \hline
\end{tabular}
\end{center}
\end{table}


