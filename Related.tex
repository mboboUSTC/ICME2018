\section{Related Work}
%
\change{Image segmentation has been studied for decades from the early time of computer vision. While comprehensive survey is out of the scope of this paper, we briefly introduce ... in this section.}


Fully convolutional networks are commonly used in many segmentation tasks \cite{Long2015,Badrinarayanan2015,Noh2015,Ronneberger2015,Chen2016a,Chen2017,Zhao2016}, which use the uppooling \cite{Badrinarayanan2015} and transposed convolutional \cite{Noh2015} to generate dense predictions.
However due to the successive downsampling layers used in FCN, the localized contours are usually poor and coarse.
To this end, \cite{Chen2016a} proposed the dilated convolution by introducing zeros into original kernel, which enlarges the perception field and reduces the usage of pooling layers.
U-net \cite{Ronneberger2015} directly concatenates the lower and higher feature maps to supplement the lost detail information by downsampling.
Recently, a multi-task framework by simultaneously implementing semantic segmentation and contour predictions is developed by DCAN \cite{Chen2017}.
They use the auxiliary contours to optimize the segmentations and obtained state-of-the-art performance in challenging medical segmentation. 
PSPNet \cite{Zhao2016} is the state of the art segmentation model in many segmentation competitions such as PASCAL VOC and cityscapes.
It utilized the powerful ResNet \cite{He2016} for feature extraction and proposed a global pyramid pooling module for better detail predictions.
In this paper, we develop a new framework to process a more challenging task, which is also important but difficult for existing methods as illustrated in Sec. \ref{sec:intro}. 

\xj{There should be another paragraph introducing contour-based methods.}