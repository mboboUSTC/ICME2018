% Template for ICME 2018 paper; to be used with:
%          spconf.sty  - ICASSP/ICIP/ICME LaTeX style file, and
%          IEEEbib.bst - IEEE bibliography style file.
% --------------------------------------------------------------------------
\documentclass{article}
\usepackage{spconf,amsmath,epsfig}

\pagestyle{empty}


\begin{document}\sloppy

% Example definitions.
% --------------------
\def\x{{\mathbf x}}
\def\L{{\cal L}}


% Title.
% ------
\title{ACCURATE SEGMENTATION OF SYNAPTIC CLEARANCE WITH DEEP CONVOLUTIONAL NETWORKS AND A NOVEL CONTOUR GROWING ALGORITHM}
%
% Single address.
% ---------------
\name{Anonymous ICME submission}
\address{}


\maketitle


%
\begin{abstract}
Synaptic cleft is an importance area for neuroscientists to analyse the macromolecular complexes related to neurotransmitter transmission.
However the high noise and low signal-to-noise ratio of raw tomogram make it great challenging to automatically segment.
In this paper, we proposed a novel contour growing algorithm combined with deep convolutional networks to localize the accurate synaptic cleft region in such high noisy images.
Starting with an initial segmentation form deep convolutional networks, our growing algorithm first finds out an initial piece of contour from the pre-segmentation, and then grows it to localize the whole synaptic cleft region.
Different from the region based methods and dense prediction networks, our model focus on localizing the plausible contours rather than internal texture, which is more robust to the noise in tomogram data.
Experiments will demonstrate the effectiveness and superiority of our proposed method on segmenting the target regions in cryo-electron tomogram data.
\end{abstract}
%
\begin{keywords}
Biomedical segmentation, deep convolutional networks, contour growing, active contour deformation,
\end{keywords}
%
%%%%%%%%% BODY TEXT
\section{Proposed Algorithm}
\label{sec:algorithm}


Our proposed algorithm consists of two steps:
a) a FCN-style network providing a coarse segmentation of synaptic cleft;
b) a contour growing algorithm that uses the coarse segmentation to localize whole contours of synaptic cleft region.
The pipeline of our framework is illustrated in Figure~\ref{fig:cg}.

\subsection{Pre-segmentation}

To accurately localize the location of our target regions, deep convolutional networks \xj{an FCN-style network is} first implemented to give a coarse segmentation for synaptic cleft.
\xj{Do we explore different networks for rough localization? How do different networks affect the finaly results? Say, is the contour-growing part strong enough to find accurate contours with initial segmentation of different level of accuracy?}
The architecture of the networks are based on famous DeepLab \cite{Chen2016a}, which uses the dilated convolution for lager reception field.
%
Differently, we change the classifier of DeepLab to be a binary classifier and use a weighted loss for mitigating unbalanced label problem in our task.
For the problem of limited data caused by expensive acquisition, the transfer learning strategy is used by fine-tuning the weights of lower layers on the off-the-shelf model from DeepLab, which has been well trained on natural images.

\subsection{Contour growing algorithm}
\subsubsection{curve evolving}
\label{sec:curve_evolving}

Based on the binary mask from pre-segmentation, we first generate an initial \change{central curve} from the mask, which will be subsequently evolved to attach to the target contour.
%
In order to make sure that the initial curve is inside the cleft region, it will be generated as short as possible, such as the green dotted line in Figure~\ref{fig:cg}.
Then the initial curve will be evolved twice to respectively attach to presynaptic and postsynaptical membranes, which are exactly the contours of synaptic cleft.

Similar to the traditional snake model \cite{Kass1988}, the initial curve is expressed by $\mathbf{v}(s)=[x(s),y(s)]$, where $s\in[0,1]$, and our goal is to minimize the following energy function:
\begin{eqnarray}\label{Eq:Etotal}
E_{total} =& \int_{0}^{1}\big[E_{int}\{\mathbf{v}(s)\}+E_{ext}\{\mathbf{v}(s)\}\big]ds\nonumber \\
E_{int} =& \int_{0}^{1}\big [\alpha\{\mathbf{v}^{'}(s)\}+\beta\{\mathbf{v}^{''}(s)\}\big]ds \\
E_{ext} =& \int_{0}^{1}\big[I\{\mathbf{v}(s)\} + \kappa G\{\mathbf{v}(s)\}\big]ds\nonumber
\end{eqnarray}
\xj{two layers of integration in above eq?}
where $\mathbf{v}^{'}(s)$ and $\mathbf{v}^{''}(s)$ are respectively the first-order and second-order derivatives of $\mathbf{v}(s)$.
$I$ is the Gaussian smoothed image.
$G$ is the gradient magnitude map.
According to \cite{Kass1988}, minimizing Eq.~\ref{Eq:Etotal} can be obtained by iteratively updating the following equations:
\begin{eqnarray}\label{Eq:GVF}
\mathbf{v}_{t+1}(s) = (A+\gamma I)^{-1}(\gamma \mathbf{v}_t(s)+\mathbf{f}\{x_t(s)\})
\end{eqnarray}
where $A$ is a matrix representing the internal tension, $\gamma$ controls the evolving rate and $\mathbf{f}=[f_{x},f_{y}]$ are the gradient maps calculated from $E_{ext}$ using Gradient Vector Flow (GVF) algorithm \cite{Xu1998}.
\xj{what is $x_t(s)$? x-coordinate? or just $s_t$?}

However Eq~\ref{Eq:GVF} is very sensitive to noisy regions, which easily trap some control points. 
Furthermore the capture range of tension $\mathbf{f}$ is usually limited \cite{Cohen1991}, making its performance depend heavily on the initial curve.
Our experiments illustrate the deficiencies of GVF in detail and show some representative cases in Fig.~\ref{fig:gvf}.
Therefore, we propose a new updating strategy by:
\begin{eqnarray}\label{Eq:update}
\mathbf{v}_{t+1}(s) = (A+\gamma I)^{-1}(\gamma \mathbf{v}_t(s)+E_{ext}\{\mathbf{v}(s)\}\mathbf{n}\{\mathbf{v}_t(s)\})
\end{eqnarray}
$\mathbf{n}\{\mathbf{v}(s)\}$ are the normal vectors of $\mathbf{v}(s)$ with consistent orientations.

In Eq.~\ref{Eq:update}, the direction of external tension are fixed as the normal direction of $\mathbf{v}_(s)$, whose magnitudes are controlled by $E_{ext}\{\mathbf{v}(s)\}$ instead of a constant value in Ballons model \cite{Cohen1991}.
The advantages of Eq.~\ref{Eq:update} are follows:
a) the capture range of external tension are much larger, due to fixed tension $\mathbf{n}\{\mathbf{v}(s)\}$;
b) it is easier to reach the global optimum than GVF, because $E_{ext}$ will soon vanish in the contour region;
c) In the noise region, once the internal tension pulls a trapped $\mathbf{v}(s)$ out, our external tension will soon push it away.
The above situations are shown in Figure~\ref{fig:gvf}.

By setting opposite normal vectors ($\mathbf{n}_{+}$ and $\mathbf{n}_{-}$ in Figure~\ref{fig:cg}), $\mathbf{v}(s)$ will be evolved along opposite direction and well attach to both presynaptic and postsynaptical membranes.
The final evolved curves are respectively denoted as $\mathbf{c}_1(s)$ and $\mathbf{c}_2(s)$.

\begin{figure}[t]
\begin{minipage}[b]{1.0\linewidth}
  \centering
 \centerline{\epsfig{figure=Figs/FigG.pdf,width=8.5cm}}
\end{minipage}
\caption{An example of contour growing formulated by Eq.~\ref{Eq:sg}.
        The yellow solid arrow indicates the growing direction of last stage.
        The dotted arrows give several candidates of next growing direction, while the one with the darkest color are preferred.
        After finding the optimal length, the longest arrow is chosen as a new piece of growing contour.}
\label{fig:g}
\end{figure}

\subsubsection{synchronous growing}
With two pieces of contour $\mathbf{c}_1(s)$ and $\mathbf{c}_2(s)$, we then grow them to localize the whole presynaptic and postsynaptic membranes.
The challenge of this part is that they should be grown correctly and synchronously to compute the exact distance between two membranes for termination judging.

First, we formulate the process of growing $\mathbf{c}_1(s)$ as finding a piece of line $\mathbf{l}(s)$ by:
\begin{eqnarray}\label{Eq:sg}
&\arg\min_{\mathbf{l}(s)} E_{int}\{\mathbf{l}(s)+c_1(1))\}+\rho(\overrightarrow{c_1}(1)*\overrightarrow{l}(0))\\
&st. \tau_1\leq (\overrightarrow{c_1}(0)*\overrightarrow{l}(0))\leq \tau_2\nonumber
\end{eqnarray}
where $c_1(1)$ is the end point of curve $\mathbf{c}_1$ and $\overrightarrow{c_1}(0)$ is the tangent vector of point $c_1(1)$.
Especially, $\mathbf{l}(s)$ is straight and can be expressed by a direction vector $\overrightarrow{l}(0))$ and a length $l$.
And $*$ defines the inner product of two vectors.
The firs term of Eq.~\ref{Eq:sg} hope $\mathbf{l}(s)$ to be inside the membrane, while the second term tends to select the growing direction following the previous growing direction.
$\rho$ is a tradeoff parameter, and $\tau_1,\tau_2$ add a hard restraint on the growing direction to be not changed too much.

Optimal solution of Eq.~\ref{Eq:sg} can be obtained by using the EM algorithm.
Explicitly, we first fixed $l$ to find an optimal $\overrightarrow{l}(0))$, and then fix $\overrightarrow{l}(0))$ to find a better $l$.
Experiments show that two times of iteration is enough for most cases to give a satisfying piece of new growing membrane as shown in Fig.~\ref{fig:g}.

Next to synchronously grow $\mathbf{c}_1(s)$ and $\mathbf{c}_2(s)$, we split the growing process into several periods and decide which curve grows in each period.
Especially, we set two variables $g_1^{t+1}$ and $g_2^{t+1}$ to determine the growing state of $\mathbf{c}_1^{t}(s)$ and $\mathbf{c}_2^{t}(s)$ at stage $t+1$ by:
\begin{eqnarray}\label{Eq:gf}
g_1^{t+1},g_2^{t+1} = \left\{\begin{array}{cc}
0,1&if \overrightarrow{c}^t_1(1)*(c_2^t(1)-c_1^t(1))\geq 0 \\
1,0&if \overrightarrow{c}^t_1(1)*(c_2^t(1)-c_1^t(0))\leq 0\\
1,1& else\\
\end{array}\right.
\end{eqnarray}
where $c_1^t(0)$, $c_1^t(1)$ and $c_2^t(1)$ are three endpoints of two growing membranes as shown in Figure~\ref{fig:sg}, of which the positions determine $g_1^{t+1}$ and $g_2^{t+1}$ ($1$ for growing and $0$ for waiting).
Different situations of Eq.~\ref{Eq:gf} are shown in Figure~\ref{fig:sg}.
The distance between two membranes can be calculated by:
\begin{eqnarray}\label{Eq:d}
d^{t+1} = \frac{||c_1^{t+1}(1)-c_2^{t+1}(1)||+ ||c_1^{t+1}(0)-c_2^{t+1}(0)||}{2}
\end{eqnarray}
Once $d^{t+1}$ is beyond the range of reasonable cleft width, the growing will be terminated.
\begin{figure}[t]
\begin{minipage}[b]{1.0\linewidth}
  \centering
 \centerline{\epsfig{figure=Figs/FigSG.pdf,width=8.5cm}}
\end{minipage}
\caption{Diagram of different situations illustrated in Eq.~\ref{Eq:gf}.
        The green line is the new growing contour, while the red line is the previous contour}
\label{fig:sg}
\end{figure}
\section{EXPERIMENTS}
\label{sec:experiments}
In this section, we will evaluate our framework with different parameters and compare our method with several state-of-the-art methods on segmenting the synaptic cleft region in electron micrographs.

~\noindent\textbf{Dataset}
Synaptic images are obtained by cryo-electron tomography (CET), from which we can directly observe a native environment of synaptic structures in a high resolution (about $1500\times 1500$).
\xj{native environment or natural environment?}
%
And localizing the accurate contour of a target region in such a crowded and natural environment is significantly challenging, due to the high nose and low signal-to-noise ratio.
In this paper, our goal is to extract the synaptic cleft region, which is adjacent to a synapse and receives neurotransmitter molecules from another synapse.
Especially, only the cleft between two synapse, whose width is about $20\sim30$ nm ($40\sim70$ pixels in our electron micrographs), might be our plausible synaptic cleft.
%
For evaluating our method, we build a synaptic electron micrographs dataset, consisting $400$ synaptic images for training and $159$ images for testing.
All the image are observed in raw resolution and labeled by biomedical experts.

~\noindent\textbf{Implementation Details}
The training strategy of our DeepLab mode follows \cite{Chen2016a}.
For such a large resolution, we crop a $321\times 321$ region \cite{Chen2016} from the original image as the input to DeepLab.
In order to avoid over fitting, the training dataset is augmented by flipping and rotation, finally containing $19200$ images.

During curve evolving, $\alpha$, $\beta$, $\kappa$ and $\gamma$ are respectively set as $0.2$, $0.2$, $0.3$ and $1$, which can give a best performance in our dataset.
For synchronous growing, $\rho$ is set as $0.25$ and $\tau_1$, $\tau_2$ are $-\frac{\sqrt{2}}{2}$, $\frac{\sqrt{2}}{2}$ to constraint the direction of a new piece of contour to deviating $[-\frac{\pi}{4},\frac{\pi}{4}]$ from last growing direction.
And when $d^{t}$ is beyond the range of $[40,80]$, the growing is terminated.

We use two metrics \cite{Cheng2017} to evaluate our method on segmentation task:
a) pixel accuracy, which evaluates the percentage of positive true pixels over the whole pixels;
b) pixel intersection-overunion (IOU) averaged across different classes.


~ \noindent\textbf{Superiority of Eq.~\ref{Eq:update}.}
In this part, we visualize the deficiencies of traditional updating Eq.~\ref{Eq:GVF} in GVF and represent the superiority of our strategy of Eq.~\ref{Eq:update}.
Firstly, in some flat regions with small gradients $\mathbf{f}$, the external tension is too weak to drive $\mathbf{v}(s)$ to move against to internal tension.
Therefore, it requires the initial curve to be away from the flat regions (Fig.~\ref{fig:gvf} (a)).
Secondly, the external tension in some noisy regions will be gyrate, which easily trap some $\mathbf{v}(s)$ (Fig.~\ref{fig:gvf} (b)).
Although sometimes the internal tension may pull the trapped point out, the gyrate $\mathbf{f}$ will trap it again.
And using Eq.~\ref{Eq:update} , once $\mathbf{v}(s)$ is pulled away from the noisy region, our external tension with fixed normal direction will soon push it away.
Thirdly, as the gradients $\mathbf{f}$ near to contours are usually large (Fig.~\ref{fig:gvf} (c)), $\mathbf{v}(s)$ easily cross the optimal positions by over-huge tension, while our external tension are controlled by $E_{ext}$, which will soon vanish in contour region and make the updating more robust.

~\noindent\textbf{Results}
We evaluate the state of the art segmentation methods, including FCN \cite{Long2015}, U-net \cite{Ronneberger2015}, DeepLab (vgg16, ResNet-101) \cite{Chen2016a} and PSPNet \cite{Zhao2016}, with our proposed model on segmenting synaptic cleft region.

Table~\ref{tab:report1} reports the pixel accuracy and mean IOU of different methods, while Fig.~ represents some visual instances of these methods.
From Table~\ref{tab:report1}, it can be found that our model gives the best performance among existing methods and our localized contours are much more precise and complete than single FCN based methods in Fig..
Especially, Unet performs better than FCN and DeepLab(vgg16), due to richer features extracted by U-shaped architecture.
The effects of pyramid pooling module used in PSPNet are not obvious compared to DeepLab(ResNet-101) in our task.
And the results of DeepLab(ResNet-101) and PSPNet are better than other FCNs, which demonstrate the effectiveness of deeper ResNet.
From Fig. , FCNs can localize the correct positions of synaptic cleft in most cases, but their contours are not precise enough for further analysis.

\begin{table}[t]
\begin{center}
\caption{Table caption} \label{tab:report1}
\begin{tabular}{|c|c|c|}
  \hline
  % after \\: \hline or \cline{col1-col2} \cline{col3-col4} ...
   & Pixel Accu. & mean IOU
  \\
  \hline
  FCN & 0.9923 & 0.5258 \\
  Unet &  0.9939 & 0.6359 \\
  DeepLab(Vgg16) & 0.9838 & 0.5867 \\
  DeepLab(ResNet-101) & 0.9951 & 0.7164 \\
  PSPNet & Cell 5 & Cell 6 \\
  Contour Growing & $\mathbf{0.9974}$ & $\mathbf{0.7848}$ \\
  \hline
\end{tabular}
\end{center}
\end{table}

Furthermore in order to prove the robustness of our model to initial segmentation, we explore the effects of different pre-segmentation model to our contour growing results.
It should be noted that DeepLab in Table~\ref{tab:report2} indicates the ResNet101 version of DeepLab, which is our default pre-segmentation network.
From the results in Table~\ref{tab:report2}, it demonstrates that our contour growing algorithm can obviously improve the results of different pre-segmentation model.
And observed from Fig.~\ref{}, as long as the correct location of synaptic cleft region is provided (Unet, DeepLab and PSPNet), our model can well localize the whole contours of target region.

\begin{table}[t]
\begin{center}
\caption{Table caption} \label{tab:report2}
\begin{tabular}{|c|c|c|}
  \hline
  % after \\: \hline or \cline{col1-col2} \cline{col3-col4} ...
   & Pixel Accu. & mean IOU
  \\
  \hline
  Contour Growing$+$FCN & 0.9956 & 0.6339 \\
  Contour Growing$+$Unet & 0.9962 & 0.7720 \\
  Contour Growing$+$DeepLab & 0.9974 & 0.7848 \\
  Contour Growing$+$PSPNet & Cell 5 & Cell 6 \\
  \hline
\end{tabular}
\end{center}
\end{table}

\section{CONCLUSION}
\label{sec:conclusion}
In this paper, we provide an effective method to segment the synaptic cleft region in challenging cryo-electron tomography data.
With the help of pre-segmenting by CNNs, our contour growing algorithm can accurately localize the whole contour of the synaptic cleft, which is more robust to noise and complex shape of target region.
Especially, a new updating strategy for contour evolving method is superior to traditional GVF, which can be extended to many tasks.
The experiments show that our framework is very effective in segmenting synaptic cleft, which is significant for biomedical analysis.

%{\small
%\bibliographystyle{ieee}
%\bibliography{egbib}
%}
% -------------------------------------------------------------------------
\bibliographystyle{IEEEbib}
\bibliography{egbib}

\end{document}
